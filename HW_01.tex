\documentclass[11pt]{article}

\usepackage{graphicx}

\title{Machine Learning, Homework 1}
\author{Ben Senator}
\date{\today}

\begin{document}

\maketitle

\textbf{Question 1.} \textit{Will the cumulative number of people who have gone to college converge or diverge?}

Under the assumption that the human population will continue to exist into the distant future, and constraining our thinking to that future, the cumulative number of people who have gone to college will diverge.

\textbf{Question 2.} \textit{The release of chlorofluorocarbons (CFCs) was banned by treaty and CFCs decay in the atmosphere over time. Would we expect the quantity of CFCs to converge after an arbitrarily long period of time without new emissions? Explain.}

We would indeed expect the quantity of CFCs to converge after an arbitrarily long period of time without new emissions. 
Specifically, we would expect the CFCs to decay to a count of 0. 
Consider that each CFC emitted affects the overall count of CFCs by $+1$ and each CFC that decays affects the overall count $-1$.
Given a limited number of CFCs and that all \textit{must} decay eventually, this series converges to 0.

\textbf{Question 3.} 3. Which series converges the fastest: $\frac{i^2-3}{i^3}$, $\frac{i^3-3}{i^5}$, or $\frac{100}{i^3}$? Explain.



\textbf{Question 4.}

\textbf{Question 5.}

\textbf{Question 6.}

\textbf{Question 7.}



\end{document}