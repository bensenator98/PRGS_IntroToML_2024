\documentclass[11pt]{article}

\usepackage{graphicx}

\title{Machine Learning, Homework 1}
\author{Ben Senator}
\date{\today}

\begin{document}

\maketitle

\textbf{Question 1.} \textit{Will the cumulative number of people who have gone to college converge or diverge?}

Under the assumption that the human population will continue to exist into the distant future, and constraining our thinking to that future, the cumulative number of people who have gone to college will diverge.

\textbf{Question 2.} \textit{The release of chlorofluorocarbons (CFCs) was banned by treaty and CFCs decay in the atmosphere over time. Would we expect the quantity of CFCs to converge after an arbitrarily long period of time without new emissions? Explain.}

We would indeed expect the quantity of CFCs to converge after an arbitrarily long period of time without new emissions. 
Specifically, we would expect the CFCs to decay to a count of 0. 
Consider that each CFC emitted affects the overall count of CFCs by $+1$ and each CFC that decays affects the overall count $-1$.
Given a limited number of CFCs and that all \textit{must} decay eventually, this series converges to 0.

\textbf{Question 3.} \textit{Which series converges the fastest: $\frac{i^2-3}{i^3}$, $\frac{i^3-3}{i^5}$, or $\frac{100}{i^3}$? Explain.}

The fastest-converging series is $\frac{100}{i^3}$ as this series has the largest ratio of denominator-heavy powers.

\textbf{Question 4.} \textit{Consider two algorithms: (A) has quadratic convergence and requires $\mathcal{O}(n^3)$ operations and (B) has linear convergence and requires $\mathcal{O}(n^2)$ operations. What would be the characteristics of a data science project when algorithm (B) would be preferable?}

As there are an exponentially increasing number of required operations in both algorithms but algorithm (B) has a lower rate of exponential growth than algorithm (A), we would desire algorithm (B) when $n$ is large.
This would save us time and computational energy.

\textbf{Question 5.} \textit{Describe whether and how the probability of rain on a given day could be described as a series with the Markov property.}

\textbf{Question 6.} \textit{Describe how representation bias could affect a facial recognition algorithm?}

\textbf{Question 7.} \textit{Describe how algorithmic practices can lead to poor outcomes and one possible approach to mitigating algorithmic bias.}



\end{document}